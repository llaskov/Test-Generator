\documentclass[10pt,a4paper,twocolumn]{article}

% packages
\usepackage{amsmath, amsthm, amssymb}
\usepackage[T1]{fontenc}
\usepackage[bitstream-charter]{mathdesign}
\usepackage{dsfont}
\usepackage{verbatim}
\usepackage[left=1.5cm, right=1.5cm, top=4cm, foot=1.5cm, headheight=0.5cm]{geometry}
\usepackage{verbatim}
\usepackage{fancyhdr}
\usepackage{enumerate}
\title{} \author{} \date{}

% body of the document
\begin{document}

% Variant1
	\thispagestyle{fancy}
	\fancyhf{}
	\fancyhead[L]{\vspace{-0.5cm}\large Name and faculty number: ....................................................................................................................
	\\
	\vspace{0.5cm} \normalsize \textbf{NETB358} \emph{Java programming, Exam session test} \textbf{2016--2017 Spring}}
\fancyhead[R]{Variant 1}
\fancyfoot[R]{Each correct answer brings 2 pt.; maximum points for the test is 40 pt.}
\begin{enumerate}
\item Which one is correct:
\begin{enumerate}[(a)]
	\item \texttt{Box<Integer> integerBox = new Box<Integer>();}
	\item \texttt{Box<Integer> integerBox = new <Integer>Box();}
	\item \texttt{Box integerBox<Integer> = new Box<Integer>();}
	\item \texttt{Box<int> integerBox = new Box<int>();}
\end{enumerate}
\item Uniform resource locator:
\begin{enumerate}[(a)]
	\item URL is a Java interface
	\item an address to a resource on the Internet
	\item a set of decimal values separated by dots that indicate an network address
	\item a combination between an IP address and a port number
\end{enumerate}
\item Java networking programs are programmed at:
\begin{enumerate}[(a)]
	\item application layer
	\item transport layer
	\item data link layer
	\item network layer
\end{enumerate}
\item Which is not true:
\begin{enumerate}[(a)]
	\item construction of parametrized types is impossible
	\item no static fields of type parameters
	\item you can create arrays of generic types
	\item type parameters cannot be primitive data types
\end{enumerate}
\item \textit{Catch or Specify Requirement} means that:
\begin{enumerate}[(a)]
	\item Java code catches the exception and then re throws it again
	\item catch exception in \texttt{try} or re throw it with \texttt{throws}
	\item JVM will handle exception by itself
	\item Java code does not have to catch the exception
\end{enumerate}
\item In datagram-based communication
\begin{enumerate}[(a)]
	\item server binds a socket to a port number
	\item port numbers are not used
	\item the packet itself contains the port number
	\item the client binds a socket to a port number
\end{enumerate}
\item A socket is:
\begin{enumerate}[(a)]
	\item a connectionless packet delivery service
	\item receiving endpoint for IP multicast packages
	\item one endpoint of a two-way communication link between two programs on the network
	\item sending/receiving endpoint for packet delivery service
\end{enumerate}
\item \begin{verbatim}
FileReader in = null;
try {
in = new FileReader ("input.txt");
}
\end{verbatim}
\begin{enumerate}[(a)]
	\item unbuffered input byte stream
	\item unbuffered input character stream
	\item buffered input character stream
	\item buffered input byte stream
\end{enumerate}
\item Swing is:
\begin{enumerate}[(a)]
	\item the Java 2D API
	\item the part of NetBeans that allow GUI development
	\item third-part library for GUI development in Java
	\item the project that comprise the lightweight GUI components in JFC
\end{enumerate}
\item Look is:
\begin{enumerate}[(a)]
	\item both how widgets appear and behave
	\item how widgets appear
	\item the accessibility API of the widget
	\item how widgets behave
\end{enumerate}
\item Swing code runs:
\begin{enumerate}[(a)]
	\item on different threads at the same time
	\item in a separate process, designed to work with GUI components
	\item on a separate thread, the event dispatch thread
	\item obligatory in the main thread
\end{enumerate}
\item Each GUI component is contained in \textit{containment hierarchy:}
\begin{enumerate}[(a)]
	\item it can be outside the containment hierarchy
	\item multiple times, depending on its occurrence
	\item two times: once in the content pane, and once in the \texttt{JPannel}
	\item only once
\end{enumerate}
\item Relational databases organize data
\begin{enumerate}[(a)]
	\item in networks
	\item in binary trees
	\item in objects
	\item in tables
\end{enumerate}
\item Database table structure:
\begin{enumerate}[(a)]
	\item the number of both rows and columns is not fixed
	\item fixed number of rows and named columns
	\item fixed number of rows and any number of named columns
	\item fixed number of named columns and any number of rows
\end{enumerate}
\item In Swing layout is provided by:
\begin{enumerate}[(a)]
	\item content pane of the top-level containers
	\item the graphical containers themselves
	\item NetBeans graphical tools
	\item layout managers applied to any graphical container
\end{enumerate}
\item The following SQL statement will:
\begin{verbatim}
CREATE TABLE Employee (
    Id CHAR(7) PRIMARY KEY,
    Name VARCHAR(10),
    Age INTEGER,
    Gender CHAR(1)
)
\end{verbatim}
\begin{enumerate}[(a)]
	\item create a table with four rows
	\item create a table with ten rows
	\item create a table with four columns
	\item create a table with seven columns
\end{enumerate}
\item Single Java application can be composed by:
\begin{enumerate}[(a)]
	\item single thread and multiple processes
	\item single process and multiple threads
	\item single process and single thread
	\item multiple processes managed by JVM
\end{enumerate}
\item To implement concurrency in Java, use:
\begin{enumerate}[(a)]
	\item processes
	\item JVM -- it does it automatically
	\item threads
	\item the event dispatch thread
\end{enumerate}
\item In a join query the involved tables are listed in
\begin{enumerate}[(a)]
	\item \texttt{FROM} clause
	\item \texttt{WHERE} clause
	\item \texttt{SELECT} clause
	\item are not listed because the syntax \texttt{TABLE.Column} is used
\end{enumerate}
\item Thread interference means:
\begin{enumerate}[(a)]
	\item two operations that run in different threads, acting on the same data, interleave
	\item different threads have inconsistent actions with the same data
	\item a thread cannot access a resource that is needed for it to complete
	\item multiple threads are waiting for each other to complete
\end{enumerate}
\end{enumerate}
\clearpage

% Variant2
	\thispagestyle{fancy}
	\fancyhf{}
	\fancyhead[L]{\vspace{-0.5cm}\large Name and faculty number: ....................................................................................................................
	\\
	\vspace{0.5cm} \normalsize \textbf{NETB358} \emph{Java programming, Exam session test} \textbf{2016--2017 Spring}}
\fancyhead[R]{Variant 2}
\fancyfoot[R]{Each correct answer brings 2 pt.; maximum points for the test is 40 pt.}
\begin{enumerate}
\item Generics in Java are a mechanism to parametrize:
\begin{enumerate}[(a)]
	\item data types
	\item methods input values
	\item method return values
	\item class identifiers
\end{enumerate}
\item \begin{verbatim}
Pair <Integer, Integer> pairInt = new Pair <>();
Pair <Number, Number> pairNumb = pairInt ;
\end{verbatim}
\begin{enumerate}[(a)]
	\item correct, because of type inference
	\item correct, \texttt{Pair <Integer, Integer>} is a subtype of \texttt{Pair <Number, Number>}
	\item not correct, \texttt{Pair <Integer, Integer>} is not a subtype of \texttt{Pair <Number, Number>}
	\item not correct, because the type arguments after the constructor are omitted
\end{enumerate}
\item Generics after compilations:
\begin{enumerate}[(a)]
	\item are replaced with the concrete values
	\item remain parametrized types
	\item are replaced with the type arguments
	\item type parameters are replaced by \texttt{Object} or with their bounds
\end{enumerate}
\item User datagram protocol:
\begin{enumerate}[(a)]
	\item connection-based protocol, provides reliable data flow
	\item protocol used for file transfer
	\item bidirectional interactive text communication
	\item protocol that sends independent packets of data
\end{enumerate}
\item Port number indicates:
\begin{enumerate}[(a)]
	\item resource location
	\item application on the computer
	\item address of the computer in Internet
	\item the protocol used
\end{enumerate}
\item In connection-based communication:
\begin{enumerate}[(a)]
	\item the packet itself contains the port number
	\item the client binds a socket to a port number
	\item server binds a socket to a port number
	\item port numbers are not used
\end{enumerate}
\item In a network communication an endpoint is defined by:
\begin{enumerate}[(a)]
	\item an IP address and a port number
	\item network interface
	\item a protocol to access, and a location
	\item a client and a server
\end{enumerate}
\item After opening a socket, a client:
\begin{enumerate}[(a)]
	\item reads/writes data to the socket
	\item closes the resources
	\item sends independent packets of data to the server
	\item opens I/O streams
\end{enumerate}
\item Which is true:
\begin{enumerate}[(a)]
	\item both Swing and JFC are part of NetBeans
	\item Swing and JFC are separate API
	\item JFC is a subset of Swing
	\item Swing is a subset of JFC
\end{enumerate}
\item Feel is:
\begin{enumerate}[(a)]
	\item the internationalization API of the widget
	\item how widgets behave
	\item both how widgets appear and behave
	\item how widgets appear
\end{enumerate}
\item Java implementation for connectionless packet delivery service:
\begin{enumerate}[(a)]
	\item \texttt{MulticastSocket}
	\item \texttt{DatagramPacket}
	\item \texttt{DatagramSocket}
	\item \texttt{Socket}
\end{enumerate}
\item Containment hierarchy is:
\begin{enumerate}[(a)]
	\item \texttt{JPannel} and its contents
	\item graphical nesting of the widgets one inside another
	\item a top-level container as root, and all components placed in it
	\item inheritance hierarchy of Swing classes
\end{enumerate}
\item SQL is
\begin{enumerate}[(a)]
	\item scripting language such as JavaScript
	\item language to control any type of DBMS
	\item programming language to query and maintain relational databases
	\item markup language to communicate with DBMS
\end{enumerate}
\item Table primary key:
\begin{enumerate}[(a)]
	\item a row that identifies the table
	\item one or more columns that uniquely identify each row
	\item a column that numbers each row
	\item a reference to a unique column in a linked table
\end{enumerate}
\item Layout manager is:
\begin{enumerate}[(a)]
	\item an option of NetBeans to position correctly graphical elements
	\item \texttt{JPannel}
	\item content pane of the top-level containers
	\item Component whose purpose is to position widgets without using distance units.
\end{enumerate}
\item The following SQL statement will:
\begin{verbatim}
SELECT Name
    FROM Employee
    WHERE Age > 30 AND GENDER = 'M'
\end{verbatim}
\begin{enumerate}[(a)]
	\item selects the names of thirty male from table  \texttt{Employee}
	\item selects the name of all employees, older than thirty, from table \texttt{Employee}
	\item selects the name of all female, older than thirty, from table \texttt{Employee}
	\item selects the name of all male, older than thirty, from table \texttt{Employee}
\end{enumerate}
\item Concurrent systems are:
\begin{enumerate}[(a)]
	\item only systems that have multiple processors
	\item systems that perform multiple tasks at the same time
	\item application that executes two things at the same time
	\item a process composed by multiple threads
\end{enumerate}
\item Which one is true:
\begin{enumerate}[(a)]
	\item a process is a lightweight thread
	\item a thread can contain multiple processes
	\item thread is a synonymous to application
	\item a process can contain multiple threads
\end{enumerate}
\item A join is a query that:
\begin{enumerate}[(a)]
	\item links tables together
	\item involves multiple columns
	\item involves multiple tables
	\item sets the foreign key for a table
\end{enumerate}
\item To avoid thread interference:
\begin{enumerate}[(a)]
	\item synchronize methods and statements
	\item sleep one of the threads
	\item synchronize constructor of the class implementing the thread
	\item interrupt one of the threads
\end{enumerate}
\end{enumerate}
\clearpage

% Variant3
	\thispagestyle{fancy}
	\fancyhf{}
	\fancyhead[L]{\vspace{-0.5cm}\large Name and faculty number: ....................................................................................................................
	\\
	\vspace{0.5cm} \normalsize \textbf{NETB358} \emph{Java programming, Exam session test} \textbf{2016--2017 Spring}}
\fancyhead[R]{Variant 3}
\fancyfoot[R]{Each correct answer brings 2 pt.; maximum points for the test is 40 pt.}
\begin{enumerate}
\item Java applet is:
\begin{enumerate}[(a)]
	\item an HTML 5 tag \texttt{<APPLET>}
	\item .java file that is executed in a web browser
	\item a JavaScript in a web page
	\item small application, written in Java, delivered in bytecode, launched from a web page
\end{enumerate}
\item Generic method is:
\begin{enumerate}[(a)]
	\item a method that has a generic type as return type
	\item any method in a generic class
	\item a method with its own type parameters
	\item a method that has a generic type as parameters
\end{enumerate}
\item \textit{Type erasure} is:
\begin{enumerate}[(a)]
	\item the ability of generics to ignore data types
	\item the mechanism in which Java implements generics
	\item the ability of a reference of a given data type to point to a object of another type
	\item the ability of Java to erase all data types
\end{enumerate}
\item Transmission control protocol:
\begin{enumerate}[(a)]
	\item protocol that sends independent packets of data
	\item bidirectional interactive text communication
	\item protocol used for file transfer
	\item connection-based protocol, provides reliable data flow
\end{enumerate}
\item Throwing an exception is:
\begin{enumerate}[(a)]
	\item is the mechanism in which the program catches the exceptional situations
	\item creating an exception object and handing it to the runtime system
	\item an error object created by the JVM
	\item error generated by the Java compiler
\end{enumerate}
\item Checked exception is an exception:
\begin{enumerate}[(a)]
	\item the application cannot recover from
	\item the application should recover from
	\item handled by the compiler
	\item external for the program
\end{enumerate}
\item \begin{verbatim}
try {

} catch (Exception e) {
    
} catch (ArithmeticException a) {
    
}
\end{verbatim}
\begin{enumerate}[(a)]
	\item you cannot have more than one \texttt{catch} for a singly \texttt{try}
	\item the types of \texttt{catch} blocks are ordered absolutely correctly
	\item the types of \texttt{catch} blocks are not ordered correctly
	\item will generate compiler warning, but will work correctly
\end{enumerate}
\item Client actions are:
\begin{enumerate}[(a)]
	\item open socket, read/write to socket,  close socket
	\item open streams, read/write to streams, close streams, open socket, close socket
	\item open socket, open streams, read/write to streams, close streams, close socket
	\item open streams, open socket, read/write to streams,  close socket, close streams
\end{enumerate}
\item JFC is:
\begin{enumerate}[(a)]
	\item the Java internationalization toolkit
	\item the Java 2D API
	\item a part of Swing toolkit
	\item broad API that contains a set of GUI components and services
\end{enumerate}
\item The order of datagrams arrival:
\begin{enumerate}[(a)]
	\item depends on how the server sends them
	\item depends on the client and how it receives them
	\item is not guaranteed
	\item is monitored by the transport layer
\end{enumerate}
\item Java implementation for sending/receiving datagram packages:
\begin{enumerate}[(a)]
	\item \texttt{Socket}
	\item \texttt{MulticastSocket}
	\item \texttt{DatagramPacket}
	\item \texttt{DatagramSocket}
\end{enumerate}
\item The three top-level container classes in Swing are:
\begin{enumerate}[(a)]
	\item \texttt{JButton}, \texttt{JCheckBox}, \texttt{JRadioButton}
	\item \texttt{JFrame}, \texttt{JDialog}, \texttt{JApplet}
	\item \texttt{JEditorPane}, \texttt{JTextPane}, \texttt{JTextField}
	\item \texttt{JMenuItem}, \texttt{JCheckBoxMenuItem}, \texttt{JToggleButton}
\end{enumerate}
\item Who is responsible for the database management:
\begin{enumerate}[(a)]
	\item SQL
	\item JVM
	\item DBMS
	\item the application that works with it
\end{enumerate}
\item Table foreign key:
\begin{enumerate}[(a)]
	\item reference to a primary key in a linked table
	\item one or more columns that uniquely identify each row
	\item a row that identifies the table
	\item a column that numbers each row
\end{enumerate}
\item Event listeners:
\begin{enumerate}[(a)]
	\item are provided by the top-level containers
	\item are part of layout managers
	\item are defined in the widget classes themselves
	\item are part of user defined classes
\end{enumerate}
\item Which column must have unique values:
\begin{verbatim}
CREATE TABLE Employee (
    Id CHAR(7) PRIMARY KEY,
    Name VARCHAR(10),
    Age INTEGER,
    Gender CHAR(1)
)
\end{verbatim}
\begin{enumerate}[(a)]
	\item fourth
	\item second
	\item first
	\item third
\end{enumerate}
\item Definition of concurrent software is:
\begin{enumerate}[(a)]
	\item refers to parallel programming algorithms
	\item an application that fights for system resources with other applications
	\item an application that performs multiple tasks at the same time
	\item computer system that perform multiple tasks at the same time
\end{enumerate}
\item An execution unit that is a self-contained environment with own resources is a:
\begin{enumerate}[(a)]
	\item process
	\item thread
	\item the whole OS
	\item both process and thread
\end{enumerate}
\item In a join query the involved columns are denoted by:
\begin{enumerate}[(a)]
	\item only the column name
	\item column name is included in the \texttt{WHERE} clause
	\item \texttt{TABLE.Column}
	\item \texttt{Column.TABLE}
\end{enumerate}
\item When \texttt{interrupt()} is invoked on a thread
\begin{enumerate}[(a)]
	\item interrupted thread sleeps and waits for a given time period
	\item the thread is terminated immediately
	\item the thread is joined with the main thread
	\item interrupted thread itself defines how to interrupt
\end{enumerate}
\end{enumerate}
\clearpage
\end{document}
